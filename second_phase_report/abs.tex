%============================= abs.tex================================
\begin{Abstract}
	
Aerodynamic shape optimization of airship envelopes generally involves obtaining the shape having lowest volumetric drag coefficient ($C_{DV}$). Low drag shapes are preferred in airships because the power/fuel consumption is directly proportional to the value of $C_{DV}$. Most existing studies related to shape optimization of airship envelopes assume them to be axisymmetric bodies of revolution. However there are many applications for which a non-axisymmetric envelope shape may be superior. For instance, stratospheric airship envelopes require a flat upper surface for increasing the power-generation capacity of solar panels mounted on them. Similarly, hybrid airships also have flatter non axi-symmetric shaped envelopes, which result in larger dynamic lift generation.  

$C_{DV}$ of axisymmetric envelopes can be estimated by performing 2D CFD analysis, but non-axisymmetric envelope shapes need 3D CFD analysis to be performed, which demands much more computational effort. If the CFD analysis code is coupled to an optimizer for obtaining shapes with low $C_{DV}$, the computational time needed to obtain a solution can be prohibitive. Hence, there is a need for resorting to an approximation scheme for estimating $C_{DV}$ as a function of envelope shape parameters. 

The present study aims at developing a Surrogate Based Design Optimization (SBDO) methodology for obtaining minimum $C_{DV}$ shapes of non axi-symmetric airship envelopes. A novel scheme for parameterization of the geometry of any 3-D envelope shape of a given volume is presented, using modified Gertler Series 58 Shape Generator. Latin Hyper-cube Sampling is used to generate trial shapes, and $C_{DV}$ for these shapes is determined by carrying out 3-D CFD analysis using OpenFOAM\textsuperscript{\textregistered}. A simple \textit{Kriging} based surrogate model is fitted through these trial points, which can predict envelope $C_{DV}$ within an acceptable level of accuracy for several randomly generated trial shapes. The envelope shape corresponding to minimum $C_{DV}$ of this surrogate function was obtained by coupling the methodology to a standard Genetic Algorithm (GA) optimizer. As expected, it is seen that the optimal shape corresponding to minimum $C_{DV}$ turns out to be an axi-symmetric body of revolution. However, if the envelope shape is constrained to be non axi-symmetric body of revolution with width atleast twice the height, then an optimum non axi-symmetric shape with $\approx 15$ \% higher $C_{DV}$ is obtained.

A multidisciplinary shape optimization problem is also attempted, in which the optimum shape is obtained by incorporating design considerations from two disciplines, viz., Aerodynamics (in the form of low  $C_{DV}$), and Structures (in the form of low von-Mises Stress $\sigma_{v}$). A composite objective function was created, in which the values of  $C_{DV}$ and $\sigma_{v}$ of the candidate envelope shape were non-dimensionalized with the values of $C_{DV}$ and $\sigma_{v}$ of the axi-symmetric shape with minimum $C_{DV}$. The optimum envelope shape was obtained that minimizes this composite objective function, with equal weightage to both the disciplines. It was found that, compared to the minimum $C_{DV}$ shape, this shape had $\approx$ 1\% higher $C_{DV}$, and $\approx 0.3$\% higher $\sigma_{v}$.
\end{Abstract}
%=======================================================================
