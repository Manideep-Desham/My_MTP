%============================= abs.tex================================
\begin{Abstract}
	
Aerodynamic shape optimization is the process of obtaining the most suitable (preferably low drag) airship envelope shape to a given problem. Multidisciplinary shape optimization means obtaining better shape subject to the trade off between performance parameters of different disciplines.

Most of the existing studies related to shape optimization of airships consider their envelopes to be axisymmetric bodies. However there are many Possibilities for non-axisymmetric envelopes.One example for the case of non-axisymmetric envelope is the case of stratospheric airships, which require an flat upper surface for making provision to mount solar panel as they are mostly powered by the solar power. Low drag shapes are preferred in airships because the power/ fuel consumption is directly dependent on drag coefficient.  The drag coefficient of axisymmetric envelopes can be estimated by performing 2D CFD analysis, but non-axisymmetric envelope shapes needs 3D CFD analysis to be performed, which demands more computational effort. 

The present study aims at developing a Surrogate Based Design Optimization (SBDO) methodology for obtaining minimum drag shapes which are non-axisymmetric. A novel scheme for geometry parameterization of envelope shapes of a given volume is presented, using modified Gertler Series 58 Shape Generator. Latin Hyper-cube Sampling is used to generate 80 shapes, and the volumetric drag coefficient ($ C_{DV} $ ) of the envelope is determined at these 80 shapes/ design points by carrying out 3-D CFD analysis using OpenFOAM\textsuperscript{\textregistered}. A simple \textit{Kriging} based surrogate model was fitted through these points, which predicted  $ C _{DV} $  values within $ 6 \% $ accuracy at 8 randomly generated trial shapes. A shape corresponding to minimum $ C _{DV} $ of this surrogate function was obtained using Genetic Algorithm (GA) optimizer. It has been found that the optimal shape for minimum drag is found to be an axisymmetric body. However if we give constraint that the body should be non-axisymmetric, the drag increased by 14.92 \%.
\end{Abstract}
%=======================================================================
