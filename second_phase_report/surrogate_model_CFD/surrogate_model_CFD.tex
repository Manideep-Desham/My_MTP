\chapter{Surrogate model for CFD}
\label{Surrogate model for CFD}
To develop a surrogate model which mimics the behaviour of CFD in finding the drag coefficient at zero degree angle of attack, it is essential to identify the limits of design parameters, perform Design Of Experiments (DOE) on them, perform CFD analysis at these DOE points and fit a surrogate model. The whole processes is described in the subsections below.

\section{Mapping design variables}
The first task before creation of DOE was the proper definition of the upper and lower limits for the six design variables (viz., $ m, r_0, r_1, C_p, \frac{l}{d} \ and \  scale \_y $). The upper and lower limits of design variables are calculated using the initial sizing done by \cite{alam2017thesis}. The design space is confined as mentioned in Table \ref{Degign space }

\begin{table}[H]
	\centering
	\caption{Design Space}
	\label{Degign space }
	%\begin{ruledtabular}
	\begin{tabular}{lll}
		\hline \hline
		Design Parameters & Min. & Max.    \\ \hline \hline
		
		$ Point\ of\ Max.\ Dia., m$ & 0.35 & 0.50     \\  
		$ Nose\ Radius, r _{o} $ & 0.20 & 0.80     \\
		$ Tail\ Radius, r _{1} $ & 0.1 & 0.5     \\  
		$ Prismatic\ Coeff., C _{p }$ & 0.55 & 0.70 \\
		$ Fineness\ Ratio, \frac{l}{d} $ &2.50 & 6.00 \\
		$Scaling\ in\ Y\ direction,\ scale\_y$ &1.00 & 5.00\\ \hline \hline
	\end{tabular}
	%	\end{ruledtabular}
\end{table}

Eighty candidate points were first obtained by Surrogates Toolbox \textbf{cite surrogate toolbox} using the Optimal Latin Hypercube Sampling (OLHS) method. The generated points were then mapped to the shapes corresponding to envelope having Reynolds number of 3.01e6.

\section{Design of Experiments study}

As mentioned previously in section \ref{DOE}, DOE study is used to extract maximum amount of information from every experiment conducted. This is especially needed when we are carrying out expensive computer or real life experiments. While developing surrogate model for CFD, each CFD analysis will take about 3 hours to complete. Hence to optimally use the computation resources, the total number experiments needs to be as low as possible but extracting maximum information about function behaviour. So, initially the number of experiments are taken as 80 and the design points are generated using the Optimal Latin Hypercube Sampling (OLHS) technique present in SURROGATES Toolbox by \cite{viana2014metamodeling}. The obtained points are given in Appendix.

\section{CFD analysis on the points obtained from DOE}
\label{Three surrogate models}

All the pre-processing required to perform CFD analysis like creating the geometry, deciding the size of domain, meshing the domain has been automated using Octave and C++ scripting. The scripts are shared in Appendix. All the routines associated with CFD like Flow parametres, Grid convergence study and the accuracy of different surrogate models are discussed in chapter \ref{Training data CFD}.







